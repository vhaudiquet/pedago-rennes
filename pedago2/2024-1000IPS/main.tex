\documentclass[12pt, a4paper, dvipsnames]{article}

\usepackage[dvipsnames, table]{xcolor}

\usepackage[french]{babel}
\usepackage{playcards}
\usepackage[T1]{fontenc}
\usepackage[a4paper]{geometry}
\geometry{top=1cm, bottom=1cm, left=1cm, right=1cm}

\usepackage{tikz}
\usepackage{pifont}
\usepackage{graphicx}


\begin{document}

\begin{center}
    \huge \textbf{1000 IPS}
\end{center}

\begin{center}
	\large {Sarah EMERY et Valentin HAUDIQUET}
\end{center}

\section{Règles du jeu}

Le jeu est inspiré du \textit{1000 bornes}. Dans ce jeu, l'objectif est de construire un processeur d'une vitesse de 1000 IPS (Instructions Par Seconde), en augmentant sa vitesse à l'aide d'extensions à celui-ci, tout en parant les attaques des adversaires qui cherchent à la faire diminuer.

\subsection{Préparation}

Au début, on mélange les cartes, puis on distribue 6 cartes à chaque joueurs.
On tire au sort un joueur qui va commencer, puis on joue dans le sens des aiguilles d'une montre.

\subsection{Tour d'un joueur}

Lors de son tour, un joueur commence par piocher une carte.
Ensuite, il peut :
\begin{itemize}
    \item Jouer une carte \textbf{extension} ; il la pose alors devant lui et la vitesse de son processeur augmente
    \item Jouer une carte \textbf{attaque} ; il choisit alors quel adversaire attaquer, pose la carte sur l'extension cible de l'adversaire, et la vitesse de son processeur diminue
    \item Jouer une carte \textbf{parade} ; il choisit alors l'attaque adaptée sur son propre processeur qu'il désire parer, et pose la parade par-dessus
\end{itemize}

Un joueur attaqué peut, s'il possède une parade adaptée, jouer alors immédiatement une carte parade, même si ce n'est pas son tour.

\subsection{But du jeu}

Dès qu'un joueur atteint une vitesse de 1000, il remporte la partie.

\section{Liste des cartes}

\begin{tabular}{|l|l|l|l|p{55mm}|}
    \hline
    \textbf{Type} & \textbf{Nom} & \textbf{Nécessite} & \textbf{Effet} & \textbf{Description} \\
    \hline
    Extension & Pipeline & N/A & +100 &
    Un pipeline permet à un processeur de réutiliser ses unités fonctionnelles en
    démarrant l'exécution d'une nouvelle instruction avant la fin de l'instruction
    précédente. Typiquement, l'exécution d'une instruction est découpée en étapes,
    comme "fetch, decode, execute", et au moment du decode de la première instruction, on peut déjà fetch la seconde.\\
    \hline
    Attaque & Aléa de contrôle & Pipeline & +50 &
    Un aléa de contrôle arrive quand un processeur pipeliné rencontre un
    branchement conditionnel (\texttt{if}). Il ne peut alors pas déterminer
    à l'avance quelle branche charger, et doit donc arrêter son pipeline jusqu'à
    la fin de la condition.\\
    \hline
    Parade & Prédicteur de branchement & Aléa de contrôle & +80 &
    Un prédicteur de branchement permet au processeur pipeliné de prédire à
    l'avance le résultat d'un branchement conditionnel (\texttt{if}), en
    utilisant l'historique relative à cette branche.\\
    \hline
    Attaque & Aléa de données & Pipeline & +50 & 
    Un aléa de donnée arrive quand un processeur pipeliné rencontre une
    instruction dont l'opérande dépend du résultat de l'instruction précédente.
    Typiquement, un aléa RAW (Read After Write) consiste en une instruction écrivant
    son résultat en mémoire, puis l'instruction suivante lisant à la même addresse.
    Le processeur doit alors arrêter son pipeline jusqu'à la fin de l'écriture.\\
    \hline
    Parade & Bypass & Aléa de données & +70 &
    Un bypass permet d'éviter une couteuse écriture mémoire lors d'un aléa de
    données, en passant directement le résultat de la première instruction à
    la seconde instruction. \\
    \hline
\end{tabular}

\begin{tabular}{|l|l|l|l|p{55mm}|}
    \hline
    \textbf{Type} & \textbf{Nom} & \textbf{Nécessite} & \textbf{Effet} & \textbf{Description} \\
    \hline
    Attaque & Aléa structurel & Pipeline & +50 &
    Un aléa structurel arrive quand un processeur pipeliné n'a plus assez d'unité fonctionnelle pour exécuter les étapes des instructions en même temps. Typiquement, si on a trop d'accès mémoire en même temps ou plusieurs opérations arithmétiques en même temps. Dans ce cas, il doit arrêter son pipeline jusqu'à la fin de l'opération.\\
    \hline
    Parade & Unité supplémentaire & Aléa structurel & +90 &
    Des unités fonctionnelles supplémentaires permettent au processeur de ne plus
    avoir d'aléa structurel, puisqu'il y a toujours une unité libre pour exécuter
    l'étape de l'instruction.\\
    \hline
    Extension & Superscalaire & Pipeline & +100 &
    Un processeur superscalaire permet d'exécuter, en profitant du pipeline, plusieurs instructions en parallèle, depuis le même flux d'instructions.\\
    \hline
    Parade & Out-of-order & Superscalaire & +80 &
    Un processeur out-of-order va réordonner les instructions pour 
    éviter les aléas.\\
    \hline
    Extension & Cache & N/A & +50 &
    Stackable 3 fois. Un cache permet d'éviter des accès mémoires couteux en
    stockant des valeurs fréquemment utilisées.\\
    \hline
    Attaque & Cache miss & Cache & +30 & ?\\
    \hline
    Parade & Prefetching & Cache miss & +40 & ?\\
    \hline
    Processeur vectoriel & & & & \\
    \hline
    Processeur multicoeur & & & x1.5 & \\
    \hline
\end{tabular}

\end{document}

